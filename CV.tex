\documentclass[10pt, letterpaper]{article}

% Packages:
\usepackage[
    ignoreheadfoot, % set margins without considering header and footer
    top=2 cm, % seperation between body and page edge from the top
    bottom=2 cm, % seperation between body and page edge from the bottom
    left=2 cm, % seperation between body and page edge from the left
    right=2 cm, % seperation between body and page edge from the right
    footskip=1.0 cm, % seperation between body and footer
    % showframe % for debugging 
]{geometry} % for adjusting page geometry
\usepackage[explicit]{titlesec} % for customizing section titles
\usepackage{tabularx} % for making tables with fixed width columns
\usepackage{array} % tabularx requires this
\usepackage[dvipsnames]{xcolor} % for coloring text
\definecolor{primaryColor}{RGB}{0, 79, 144} % define primary color
\usepackage{enumitem} % for customizing lists
\usepackage{fontawesome5} % for using icons
\usepackage{amsmath} % for math
\usepackage[
    pdftitle={John Doe's CV},
    pdfauthor={John Doe},
    pdfcreator={LaTeX with RenderCV},
    colorlinks=true,
    urlcolor=primaryColor
]{hyperref} % for links, metadata and bookmarks
\usepackage[pscoord]{eso-pic} % for floating text on the page
\usepackage{calc} % for calculating lengths
\usepackage{bookmark} % for bookmarks
\usepackage{lastpage} % for getting the total number of pages
\usepackage{changepage} % for one column entries (adjustwidth environment)
\usepackage{paracol} % for two and three column entries
\usepackage{ifthen} % for conditional statements
\usepackage{needspace} % for avoiding page brake right after the section title
\usepackage{iftex} % check if engine is pdflatex, xetex or luatex

% Ensure that generate pdf is machine readable/ATS parsable:
\ifPDFTeX
    \input{glyphtounicode}
    \pdfgentounicode=1
    \usepackage[T1]{fontenc}
    \usepackage[utf8]{inputenc}
    \usepackage{lmodern}
\fi

\usepackage[default, type1]{sourcesanspro} 

% Some settings:
\AtBeginEnvironment{adjustwidth}{\partopsep0pt} % remove space before adjustwidth environment
\pagestyle{empty} % no header or footer
\setcounter{secnumdepth}{0} % no section numbering
\setlength{\parindent}{0pt} % no indentation
\setlength{\topskip}{0pt} % no top skip
\setlength{\columnsep}{0.15cm} % set column seperation
\makeatletter
\let\ps@customFooterStyle\ps@plain % Copy the plain style to customFooterStyle
%\patchcmd{\ps@customFooterStyle}{}{\color{gray}\textit{\small Page \thepage{} of \pageref*{LastPage}}}%{}{} % replace number by desired string
\makeatother
\pagestyle{customFooterStyle}

\titleformat{\section}{
    % avoid page braking right after the section title
    \needspace{4\baselineskip}
    % make the font size of the section title large and color it with the primary color
    \Large\color{primaryColor}
}{
}{
}{
    % print bold title, give 0.15 cm space and draw a line of 0.8 pt thickness
    % from the end of the title to the end of the body
    \textbf{#1}\hspace{0.15cm}\titlerule[0.8pt]\hspace{-0.1cm}
}[] % section title formatting

\titlespacing{\section}{
    % left space:
    -1pt
}{
    % top space:
    0.3 cm
}{
    % bottom space:
    0.2 cm
} % section title spacing

% \renewcommand\labelitemi{$\vcenter{\hbox{\small$\bullet$}}$} % custom bullet points
\newenvironment{highlights}{
    \begin{itemize}[
        topsep=0.10 cm,
        parsep=0.10 cm,
        partopsep=0pt,
        itemsep=0pt,
        leftmargin=0.4 cm + 10pt
    ]
}{
    \end{itemize}
} % new environment for highlights

\newenvironment{highlightsforbulletentries}{
    \begin{itemize}[
        topsep=0.10 cm,
        parsep=0.10 cm,
        partopsep=0pt,
        itemsep=0pt,
        leftmargin=10pt
    ]
}{
    \end{itemize}
} % new environment for highlights for bullet entries


\newenvironment{onecolentry}{
    \begin{adjustwidth}{
        0.2 cm + 0.00001 cm
    }{
        0.2 cm + 0.00001 cm
    }
}{
    \end{adjustwidth}
} % new environment for one column entries

\newenvironment{twocolentry}[2][]{
    \onecolentry
    \def\secondColumn{#2}
    \setcolumnwidth{\fill, 4.5 cm}
    \begin{paracol}{2}
}{
    \switchcolumn \raggedleft \secondColumn
    \end{paracol}
    \endonecolentry
} % new environment for two column entries

\newenvironment{threecolentry}[3][]{
    \onecolentry
    \def\thirdColumn{#3}
    \setcolumnwidth{1 cm, \fill, 4.5 cm}
    \begin{paracol}{3}
    {\raggedright #2} \switchcolumn
}{
    \switchcolumn \raggedleft \thirdColumn
    \end{paracol}
    \endonecolentry
} % new environment for three column entries

\newenvironment{header}{
    \setlength{\topsep}{0pt}\par\kern\topsep\centering\color{primaryColor}\linespread{1.5}
}{
    \par\kern\topsep
} % new environment for the header

\newcommand{\placelastupdatedtext}{% \placetextbox{<horizontal pos>}{<vertical pos>}{<stuff>}
  \AddToShipoutPictureFG*{% Add <stuff> to current page foreground
    \put(
        \LenToUnit{\paperwidth-2 cm-0.2 cm+0.05cm},
        \LenToUnit{\paperheight-1.0 cm}
    ){\vtop{{\null}\makebox[0pt][c]{
        \small\color{gray}\textit{Last updated in September 2024}\hspace{\widthof{Last updated in September 2024}}
    }}}%
  }%
}%

% save the original href command in a new command:
\let\hrefWithoutArrow\href

% new command for external links:
\renewcommand{\href}[2]{\hrefWithoutArrow{#1}{\ifthenelse{\equal{#2}{}}{ }{#2 }\raisebox{.15ex}{\footnotesize \faExternalLink*}}}


\begin{document}
    \newcommand{\AND}{\unskip
        \cleaders\copy\ANDbox\hskip\wd\ANDbox
        \ignorespaces
    }
    \newsavebox\ANDbox
    \sbox\ANDbox{}

    \placelastupdatedtext
    \begin{header}
        \fontsize{30 pt}{30 pt}
        \textbf{Tunan Wang}

        \vspace{0.3 cm}

        \normalsize
        \mbox{{\footnotesize\faMapMarker*}\hspace*{0.13cm}4200 University Ave., 402, Madison, WI 53705}%
        \kern 0.25 cm%
        \AND%
        \kern 0.25 cm%
        \mbox{\hrefWithoutArrow{mailto:tunan.wang@wisc.edu}{{\footnotesize\faEnvelope[regular]}\hspace*{0.13cm}tunan.wang@wisc.edu}}%
        \kern 0.25 cm%
        \AND%
        \kern 0.25 cm%
        \mbox{\hrefWithoutArrow{tel:+1-734-492-2371}{{\footnotesize\faPhone*}\hspace*{0.13cm}+1 734-492-2371}}%
        \kern 0.25 cm%
        \AND%
        \kern 0.25 cm%
        \mbox{\hrefWithoutArrow{https://tunan-wang.netlify.app/}{{\footnotesize\faLink}\hspace*{0.13cm}Personal website}}%
        \kern 0.25 cm%
        \AND%
     %    \kern 0.25 cm%
     %    \mbox{\hrefWithoutArrow{https://linkedin.com/in/yourusername}{{\footnotesize\faLinkedinIn}\hspace*{0.13cm}yourusername}}%
     %    \kern 0.25 cm%
     %    \AND%
        \kern 0.25 cm%
        \mbox{\hrefWithoutArrow{https://github.com/Tuna2222}{{\footnotesize\faGithub}\hspace*{0.13cm}Tuna2222}}%
    \end{header}

    \vspace{0.3 cm - 0.3 cm}


    % \section{Welcome to RenderCV!}



        
    %     \begin{onecolentry}
    %         \href{https://rendercv.com}{RenderCV} is a LaTeX-based CV/resume version-control and maintenance app. It allows you to create a high-quality CV or resume as a PDF file from a YAML file, with \textbf{Markdown syntax support} and \textbf{complete control over the LaTeX code}.
    %     \end{onecolentry}

    %     \vspace{0.2 cm}

    %     \begin{onecolentry}
    %         The boilerplate content was inspired by \href{https://github.com/dnl-blkv/mcdowell-cv}{Gayle McDowell}.
    %     \end{onecolentry}


    
    % \section{Quick Guide}

    % \begin{onecolentry}
    %     \begin{highlightsforbulletentries}


    %     \item Each section title is arbitrary and each section contains a list of entries.

    %     \item There are 7 unique entry types: \textit{BulletEntry}, \textit{TextEntry}, \textit{EducationEntry}, \textit{ExperienceEntry}, \textit{NormalEntry}, \textit{PublicationEntry}, and \textit{OneLineEntry}.

    %     \item Select a section title, pick an entry type, and start writing your section!

    %     \item \href{https://docs.rendercv.com/user_guide/}{Here}, you can find a comprehensive user guide for RenderCV.


    %     \end{highlightsforbulletentries}
    % \end{onecolentry}

    \section{Education}



        
        \begin{threecolentry}{\textbf{BS}}{
            09/2016 - 06/2020
        }
            \textbf{Shandong University}
            \begin{highlights}
                \item Bachelor of Science, major in Statistics, School of Mathematics
            \end{highlights}
        \end{threecolentry}

        \begin{threecolentry}{\textbf{MS}}{
            08/2021 - 04/2023
        }
            \textbf{University of Michigan}
            \begin{highlights}
                \item Master of Science, major in Applied Statistics, Department of Statistics
            \end{highlights}
        \end{threecolentry}

        \begin{threecolentry}{\textbf{Ph.D.}}{
            08/2023 - 
        }
            \textbf{University of Wisconsin-Madison}
            \begin{highlights}
                \item Major in Computer Science, School of Computer, Data \& Information Sciences
            \end{highlights}
        \end{threecolentry}


    
    \section{Research \& Internship Experience}



        
        \begin{twocolentry}{
            Madison, WI

            U.S.

            12/2023 - 
        }
            \textbf{University of Wisconsin-Madison}, advised by Prof. Vivak Patel
            \begin{highlights}
                \item I am investigating the operational mechanisms of non-smooth conditions
                within conventional optimization methods in deep learning. To be specific, the current gradient 
                calculation can be wrong at certain non-smooth points, our work is to come up with a stretegy to 
                fix these errors.
                \item Understanding current optimization techniques used in estimating parameters for large-scale
                Gaussian Mixture Models and innovating by developing stochastic adaptations of pertinent
                algorithms. 
            \end{highlights}
        \end{twocolentry}


        \vspace{0.2 cm}

        \begin{twocolentry}{
            Ann Arbor, MI

            U.S.

            07/2022 -
        }
            \textbf{University of Michigan}, advised by Prof. Albert S. Berahas
            \begin{highlights}
                \item Developing an adaptive method on distributed optimization settings. To be specific, we are
                developing a kind of stochastic variant (w.r.t. constraints) of SQP (sequential quadratic programming)
                which can solve distributed problems without manually setting any hyperparameters.
            \end{highlights}
        \end{twocolentry}

        \begin{twocolentry}{
            Tianjin

            China

            09/2020 - 03/2021
        }
            \textbf{Tianjin University}, Part-time RA in the associate professor Runliang Dou's group
            \begin{highlights}
                \item Research on the topic of operation research and management, take part in the Industrial Big
                Data Competition on the topic of “prediction of reservoir’s flow of hydropower station”.
            \end{highlights}
        \end{twocolentry}

        \begin{twocolentry}{
            ShenZhen, Guangzhou

            China

            07/15/2019 - 09/13/2019
        }
            \textbf{First Capital Securities Co., Ltd.}, Summer intern at Company Business Department
            \begin{highlights}
                \item Prepared the reception meeting, recorded the conversations with customers, made systematic
                investigation and evaluation reports based on the company's public information, sorted out the
                indicators of real estate companies according to the rating report of rating company, wrote internal
                rating report of the company and daily bond price changes report
            \end{highlights}
        \end{twocolentry}


        \begin{twocolentry}{
            Jinan, Shandong

            China

            05/2018 - 05/2019
        }
            \textbf{Shandong University}, Undergraduate Science and Technology Innovation Fund Project
            \begin{highlights}
                \item Group leader of 4-person research group
                \item Used traditional N-W Kernel Regression to process the data with an outlier, then added L1, 
                L2 Regularization and regularized Huber loss function to the original Algorithm to make the outcome
                with higher robustness and alleviate the overfitting problem. The results show that the curve draw by
                the Algorithm which used Huber loss function and regularized with L1 regulation has better sparsity
                and runs faster than the others
            \end{highlights}
        \end{twocolentry}


        \begin{twocolentry}{
            Beijing

            China

            07/16/2018 - 08/31/2018
        }
            \textbf{China Securities Co., Ltd.}, Summer intern at Fixed Income Department
            \begin{highlights}
                \item Collected data information of bond market and prepared summary report; assisted to
                complete special reports and in-depth reports on credit debts; prepared meeting reports and bond
                issuance notice, drafted contracts, etc.
            \end{highlights}
        \end{twocolentry}



    
    % \section{Publications}



        
    %     \begin{samepage}
    %         \begin{twocolentry}{
    %             Jan 2004
    %         }
    %             \textbf{3D Finite Element Analysis of No-Insulation Coils}

    %             \vspace{0.10 cm}

    %             \mbox{Frodo Baggins}, \mbox{\textbf{\textit{John Doe}}}, \mbox{Samwise Gamgee}
    %             \vspace{0.10 cm}

    %     \href{https://doi.org/10.1109/TASC.2023.3340648}{10.1109/TASC.2023.3340648}
    %         \end{twocolentry}
    %     \end{samepage}


    
    \section{Projects}



        
        \begin{twocolentry}{
            \href{https://github.com/sinaatalay/rendercv}{PertGD}
        }
            \textbf{R package \texttt{PertGD}}
            \begin{highlights}
                \item We implement four gradient-based methods which can escape from saddle point quickly. 
                They are “Perturbed Gradient Descent”, “Perturbed Accelerated Gradient Descent”, 
                “Faster Perturbed Gradient Descent”, 
                and “Faster Perturbed Accelerated Gradient Descent”.
                \item Role: Developer
                \item Language Used: R
            \end{highlights}
        \end{twocolentry}


        \vspace{0.2 cm}

        \begin{twocolentry}{
            \href{https://github.com/numlinalg/RLinearAlgebra.jl}{RLinearAlgebra.jl}
        }
            \textbf{Julia package \texttt{RLinearAlgebra}}
            \begin{highlights}
                \item 
                \item Role: Maintainer \& Developer
                \item Language Used: Julia
            \end{highlights}
        \end{twocolentry}


        \vspace{0.2 cm}

        \begin{twocolentry}{
            \href{https://github.com/sinaatalay/rendercv}{github.com/name/repo}
        }
            \textbf{Course project optimization}
            \begin{highlights}
                \item Understanding popular optimization tools
                \item Language Used: Matlab
            \end{highlights}
        \end{twocolentry}



    \section{Teaching}
    \begin{onecolentry}
        \textbf{Teaching assistant}, University of Wisconsin-Madison
        \begin{highlights}
            \item CS 220, Data Science Programming I (Fall 2023, Spring 2024)
            \item CS 320, Data Science Programming II (Fall 2024)
        \end{highlights}


    \end{onecolentry}



    
    \section{Skills}



        
        \begin{onecolentry}
            \textbf{Languages:} Python, Julia, Matlab, R
        \end{onecolentry}

        \vspace{0.2 cm}

        \begin{onecolentry}
            \textbf{Technologies:} Optimization algorithm developer
        \end{onecolentry}

    
        {\color{red} 
        \begin{itemize}
            \item Resume explain the idea easily and address the social importance.
            \item Do not use too long sentance and seperate the bulletpoint into several different ones.
            \item DO not use I/me/we, just use verb.
            \item Combine two courses and explain the course, expand.
            \item Expand the sections, maybe add one section of Language.
        \end{itemize}
         
        }
\end{document}